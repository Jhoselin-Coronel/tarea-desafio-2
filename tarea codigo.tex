\documentclass{article}
\usepackage[spanish]{babel}
\usepackage{amsmath,amsthm,amsfonts}
\usepackage{latexsym,amssymb}
\usepackage{pifont}
\newtheorem{teor}{Teorema}
\title{MEDIDAS Y TEOREMA}
\author{Jhoselin Angélica Coronel Condori}
\date{Abril 2024}
\begin{document}
\maketitle
\section{Medidas}
\begin{dinglist}{219}
\item Por espacio medible entendemos un par ordenado $(\Omega, \mathcal{B})$ que consta de un conjunto $\Omega$ y una $\sigma$-álgebra $\mathcal{B}$ de subconjuntos de $\Omega$. Un subconjunto $A$ de $\Omega$ se llama medible si $A \in \mathcal{B}$.
\item Una medida $\mu$ en un espacio medible $(\Omega, \mathcal{B})$ es una función $\mu: \mathcal{B} \rightarrow [0, \infty]$ que satisface:
\begin{align*}
\mu(\emptyset) &= 0 \\
\mu\left(\bigcup_{i}^{\infty} E_{i}\right) &= \sum_{i}^{\infty} \mu(E_{i})
\end{align*}
para cualquier sucesión $\{E_{i}\}$ de conjuntos medibles disjuntos, es decir, $E_{i} \cap E_{j} = \emptyset$, $E_{i} \in \mathcal{B}$, $i \neq j$.
\item $(\Omega, \mathcal{B}, \mu)$ se llama espacio de medida.
 \end{dinglist} 
\section{Teorema}
 \textbf{Teorema} Las siguientes afirmaciones son equivalentes para un grupo $G$:
\begin{align*}
1. & \quad P(G) = 1 & 4. & \quad G' = \{1\} \\
2. & \quad G \text{ es abeliano} & 5. & \quad CG(a) = G \text{ para todo } a \in G \\  
3. & \quad Z(G) = G & 6. & \quad G/G' \cong G 
\end{align*}

\begin{proof}
   Si $P(G) = 1$, entonces $|L(G)| = |G|^2$. Luego, $L(G) = G^2$, lo cual implica que $xy = yx$ para todo $x, y \in G$. Así, $G$ es un grupo abeliano. Es inmediato observar que el razonamiento inverso también es cierto, lo que prueba que 1 es equivalente a 2.
\end{proof}
Según este resultado, para tener grados de conmutatividad diferentes de 1, debemos analizar grupos no abelianos. 
\end{document}
